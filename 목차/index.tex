% Options for packages loaded elsewhere
\PassOptionsToPackage{unicode}{hyperref}
\PassOptionsToPackage{hyphens}{url}
\PassOptionsToPackage{dvipsnames,svgnames,x11names}{xcolor}
%
\documentclass[
  letterpaper,
  DIV=11,
  numbers=noendperiod]{scrreprt}

\usepackage{amsmath,amssymb}
\usepackage{iftex}
\ifPDFTeX
  \usepackage[T1]{fontenc}
  \usepackage[utf8]{inputenc}
  \usepackage{textcomp} % provide euro and other symbols
\else % if luatex or xetex
  \usepackage{unicode-math}
  \defaultfontfeatures{Scale=MatchLowercase}
  \defaultfontfeatures[\rmfamily]{Ligatures=TeX,Scale=1}
\fi
\usepackage{lmodern}
\ifPDFTeX\else  
    % xetex/luatex font selection
\fi
% Use upquote if available, for straight quotes in verbatim environments
\IfFileExists{upquote.sty}{\usepackage{upquote}}{}
\IfFileExists{microtype.sty}{% use microtype if available
  \usepackage[]{microtype}
  \UseMicrotypeSet[protrusion]{basicmath} % disable protrusion for tt fonts
}{}
\makeatletter
\@ifundefined{KOMAClassName}{% if non-KOMA class
  \IfFileExists{parskip.sty}{%
    \usepackage{parskip}
  }{% else
    \setlength{\parindent}{0pt}
    \setlength{\parskip}{6pt plus 2pt minus 1pt}}
}{% if KOMA class
  \KOMAoptions{parskip=half}}
\makeatother
\usepackage{xcolor}
\setlength{\emergencystretch}{3em} % prevent overfull lines
\setcounter{secnumdepth}{5}
% Make \paragraph and \subparagraph free-standing
\makeatletter
\ifx\paragraph\undefined\else
  \let\oldparagraph\paragraph
  \renewcommand{\paragraph}{
    \@ifstar
      \xxxParagraphStar
      \xxxParagraphNoStar
  }
  \newcommand{\xxxParagraphStar}[1]{\oldparagraph*{#1}\mbox{}}
  \newcommand{\xxxParagraphNoStar}[1]{\oldparagraph{#1}\mbox{}}
\fi
\ifx\subparagraph\undefined\else
  \let\oldsubparagraph\subparagraph
  \renewcommand{\subparagraph}{
    \@ifstar
      \xxxSubParagraphStar
      \xxxSubParagraphNoStar
  }
  \newcommand{\xxxSubParagraphStar}[1]{\oldsubparagraph*{#1}\mbox{}}
  \newcommand{\xxxSubParagraphNoStar}[1]{\oldsubparagraph{#1}\mbox{}}
\fi
\makeatother


\providecommand{\tightlist}{%
  \setlength{\itemsep}{0pt}\setlength{\parskip}{0pt}}\usepackage{longtable,booktabs,array}
\usepackage{calc} % for calculating minipage widths
% Correct order of tables after \paragraph or \subparagraph
\usepackage{etoolbox}
\makeatletter
\patchcmd\longtable{\par}{\if@noskipsec\mbox{}\fi\par}{}{}
\makeatother
% Allow footnotes in longtable head/foot
\IfFileExists{footnotehyper.sty}{\usepackage{footnotehyper}}{\usepackage{footnote}}
\makesavenoteenv{longtable}
\usepackage{graphicx}
\makeatletter
\def\maxwidth{\ifdim\Gin@nat@width>\linewidth\linewidth\else\Gin@nat@width\fi}
\def\maxheight{\ifdim\Gin@nat@height>\textheight\textheight\else\Gin@nat@height\fi}
\makeatother
% Scale images if necessary, so that they will not overflow the page
% margins by default, and it is still possible to overwrite the defaults
% using explicit options in \includegraphics[width, height, ...]{}
\setkeys{Gin}{width=\maxwidth,height=\maxheight,keepaspectratio}
% Set default figure placement to htbp
\makeatletter
\def\fps@figure{htbp}
\makeatother
% definitions for citeproc citations
\NewDocumentCommand\citeproctext{}{}
\NewDocumentCommand\citeproc{mm}{%
  \begingroup\def\citeproctext{#2}\cite{#1}\endgroup}
\makeatletter
 % allow citations to break across lines
 \let\@cite@ofmt\@firstofone
 % avoid brackets around text for \cite:
 \def\@biblabel#1{}
 \def\@cite#1#2{{#1\if@tempswa , #2\fi}}
\makeatother
\newlength{\cslhangindent}
\setlength{\cslhangindent}{1.5em}
\newlength{\csllabelwidth}
\setlength{\csllabelwidth}{3em}
\newenvironment{CSLReferences}[2] % #1 hanging-indent, #2 entry-spacing
 {\begin{list}{}{%
  \setlength{\itemindent}{0pt}
  \setlength{\leftmargin}{0pt}
  \setlength{\parsep}{0pt}
  % turn on hanging indent if param 1 is 1
  \ifodd #1
   \setlength{\leftmargin}{\cslhangindent}
   \setlength{\itemindent}{-1\cslhangindent}
  \fi
  % set entry spacing
  \setlength{\itemsep}{#2\baselineskip}}}
 {\end{list}}
\usepackage{calc}
\newcommand{\CSLBlock}[1]{\hfill\break\parbox[t]{\linewidth}{\strut\ignorespaces#1\strut}}
\newcommand{\CSLLeftMargin}[1]{\parbox[t]{\csllabelwidth}{\strut#1\strut}}
\newcommand{\CSLRightInline}[1]{\parbox[t]{\linewidth - \csllabelwidth}{\strut#1\strut}}
\newcommand{\CSLIndent}[1]{\hspace{\cslhangindent}#1}

\KOMAoption{captions}{tableheading}
\makeatletter
\@ifpackageloaded{bookmark}{}{\usepackage{bookmark}}
\makeatother
\makeatletter
\@ifpackageloaded{caption}{}{\usepackage{caption}}
\AtBeginDocument{%
\ifdefined\contentsname
  \renewcommand*\contentsname{목차}
\else
  \newcommand\contentsname{목차}
\fi
\ifdefined\listfigurename
  \renewcommand*\listfigurename{그림 목록}
\else
  \newcommand\listfigurename{그림 목록}
\fi
\ifdefined\listtablename
  \renewcommand*\listtablename{표 목록}
\else
  \newcommand\listtablename{표 목록}
\fi
\ifdefined\figurename
  \renewcommand*\figurename{그림}
\else
  \newcommand\figurename{그림}
\fi
\ifdefined\tablename
  \renewcommand*\tablename{표}
\else
  \newcommand\tablename{표}
\fi
}
\@ifpackageloaded{float}{}{\usepackage{float}}
\floatstyle{ruled}
\@ifundefined{c@chapter}{\newfloat{codelisting}{h}{lop}}{\newfloat{codelisting}{h}{lop}[chapter]}
\floatname{codelisting}{목록}
\newcommand*\listoflistings{\listof{codelisting}{코드 목록}}
\makeatother
\makeatletter
\makeatother
\makeatletter
\@ifpackageloaded{caption}{}{\usepackage{caption}}
\@ifpackageloaded{subcaption}{}{\usepackage{subcaption}}
\makeatother

\ifLuaTeX
\usepackage[bidi=basic]{babel}
\else
\usepackage[bidi=default]{babel}
\fi
\babelprovide[main,import]{korean}
% get rid of language-specific shorthands (see #6817):
\let\LanguageShortHands\languageshorthands
\def\languageshorthands#1{}
\ifLuaTeX
  \usepackage{selnolig}  % disable illegal ligatures
\fi
\usepackage{bookmark}

\IfFileExists{xurl.sty}{\usepackage{xurl}}{} % add URL line breaks if available
\urlstyle{same} % disable monospaced font for URLs
\hypersetup{
  pdftitle={bitPublish를 이용하여 한글 책 조판하기},
  pdflang={ko-KR},
  colorlinks=true,
  linkcolor={blue},
  filecolor={Maroon},
  citecolor={Blue},
  urlcolor={Blue},
  pdfcreator={LaTeX via pandoc}}


\title{bitPublish를 이용하여 한글 책 조판하기}
\author{}
\date{}

\begin{document}
\maketitle

\renewcommand*\contentsname{목차}
{
\hypersetup{linkcolor=}
\setcounter{tocdepth}{2}
\tableofcontents
}

\bookmarksetup{startatroot}

\chapter{머릿말: 이 책에서 다루는 것과 다루지 않는
것}\label{uxba38uxb9bfuxb9d0-uxc774-uxcc45uxc5d0uxc11c-uxb2e4uxb8e8uxb294-uxac83uxacfc-uxb2e4uxb8e8uxc9c0-uxc54auxb294-uxac83}

안녕하세요. 데이터 기반 디자인과 A/B 테스팅의 세계에 방문하신 것을
환영합니다!

저는 여러분에게 이 책의 내용들을 경험할 수 있도록 안내할 홍익대
디자인컨버전스 학부 교수 이현진입니다. 이 책은 우리 학부의 3학년 전공
수업인 UX Design(2) 수업의 3년간 진행 경험을 기반으로 집필되었고, 앞으로
해당 수업에서 교재로 사용될 목적으로 제작했습니다. 이 책의 독자가 해당
수업을 수강하는 우리 학부 학생일 수도 있고, 관심 주제가 같은 다른 학교
디자인 전공 학생일 수도 있으며, 혹은 경험있는 디자인 실무자이거나
디자이너와 일하는 인접 분야의 전문가일 수도 있을 것으로 생각하고, 이
책에서 다루는 내용의 범위와 학습 내용에 대하여 먼저 안내를 드리고자
합니다.

이 책을 수업 교재로 사용하는 기준 독자는 UX 디자인 분야에 관심을 가지고,
디자인 방법론을 학습하고자 하는 학부 3학년 학생들입니다. 이 학생들은
권장 선수 과목 수업(UX Design(1))에서 더블 다이아몬드 모델 기반의 디자인
프로세스와 대면 리서치 형식의 사용자 관찰 및 인터뷰(Contexual Inquiry
Interview)를 수행한 경험이 있고, 어피니티 다이어그램(Affinity Diagram)
기법으로 디자인 문제의 구조를 구축해봤으며, 디자인 콘셉트를 디지털
디바이스 (모바일 폰, 패드, 또는 여러 유형의 화면 디스플레이)의 GUI
디자인에 적용한 과제를 수행해 본 경험이 있습니다. 또한 일부 학생들은
이전 학기에 데이터 문해력 수업(Big Data)을 통하여 R 프로그래밍 기반의
데이터 분석 기법을 공부한 경험이 있기도 합니다. 그러나 이러한 선수 과목
수강이 필수 요건이 아니기 때문에 수강 인원의 절반 이상은 기초적인 그래픽
디자인과 인터렉션 디자인 관련 과목의 수강 경험 정도를 가지고 수업에
입문합니다. 그래서 이 수업에서는 디자인 프로세스 기초를 리뷰하는 시간도
갖고, 데이터 분석 경험이 없는 학생들을 위한 학습 경로도 제시합니다. 다만
독자 분들이 이상에서 설명한 선행 학습 주제에 대한 이해가 있으시다면 이
책의 핵심 내용을 더 빠르게 습득하실 수 있고, 만약 그렇지 않은 상태라면,
때에 따라서는 주제를 이해하는 데 필요한 기본 개념 들에 대하여 시간을
들여 보강해 가면서 공부하시면 좋겠다는 조언을 드립니다.

\section{이 책에서 다루는
것:}\label{uxc774-uxcc45uxc5d0uxc11c-uxb2e4uxb8e8uxb294-uxac83}

\begin{itemize}
\tightlist
\item
  모바일 플랫폼의 디지털미디어 서비스 디자인
\end{itemize}

이 책에서 주로 다루는 디자인 대상은 주로 모바일 폰과 같이 화면을 갖는
디지털 디바이스를 사용하는 서비스들 입니다. 플랫폼은 패드나 와치(Watch)
같이 다양한 크기일 수 도 있고, 웹 기반 또는 여러 모바일 OS를 기반으로 할
수도 있으나, 대부분 모바일 디지털 서비스들의 디자인을 대상으로 합니다.
책에서 다루는 디자인과 테스팅 방법들은 모바일 디바이스 외에 다른
플랫폼이나 서비스 형태에 적용하는 것도 가능하지만 학생들과 교수자의 관심
플랫폼이 모바일 서비스여서 수업에서는 주로 모바일 서비스를 디자인
대상으로 해왔습니다. 그래서 대부분의 예제는 모바일 앱을 사용하고
있습니다만, 학습 내용이 모바일 서비스에만 국한되는 것은 아니므로, 다른
제품 군이나 서비스를 대상으로 적용해도 문제는 없습니다.

\begin{itemize}
\tightlist
\item
  데이터 기반 디자인 (Data Driven Design)
\end{itemize}

이 책의 주제는 디자인 과정에서 데이터, 특히 정량 데이터들을 활용하는
방법입니다. 서비스 플랫폼을 통하여 자동적으로 수집되는 서비스 로그
데이터나 사용자들을 대상으로 서비스에 대한 경험이나 의견을 묻는 설문
데이터를 분석하여 디자인에 대한 의사 결정을 하고, 수행한 의사 결정이
서비스의 목표에 부합하는 개선을 이루었는지를 통계적 방법으로 검증하는
방법을 학습합니다. 서비스의 실무 담당자가 아닌 학부생으로서 이런 문제를
다루는 데는 여러 제약이 존재하지만, 한정된 여건 속에서 데이터 기반
디자인이 수행되는 원리를 이해하고 실습해보는 경험을 가지며, 나아가 이
경험을 바탕으로 실무 디자인 상황에서 빠르게 데이터 기반 디자인 수행
능력을 갖추도록 하는 것이 본 교재의 목표입니다.

\begin{itemize}
\tightlist
\item
  온라인 설문 조사 기법
\end{itemize}

데이터 기반 디자인을 위하여 설문 조사가 꼭 필요한 것은 아니고, 사용자
로그 데이터가 더 유용한 경우가 많지만, 실제 운영 중인 서비스의 데이터에
접근이 어려운 학부 수업의 상황을 고려하여 온라인 설문 조사를 통하여
서비스의 개선 디자인 검증 평가(A/B 테스팅)를 해오고 있습니다. 이를
위하여 다량의 정량 데이터 생성과 분석이 가능한 온라인 설문 수행 경험을
하고, 설문에서 분석한 주요 서비스 지표(KPI) 값들을 서비스의 디자인
해결안과 연결하여 사고하는 연습을 진행합니다.

\begin{itemize}
\tightlist
\item
  린(Lean) 디자인 프로세스와 A/B 테스팅
\end{itemize}

교재에서 기반으로 하는 디자인 프로세스는 린(Lean) 디자인 프로세스
입니다. 이것은 데이터에 의한 정량적 가설 검증을 디자인 의사결정의 핵심
방법론으로 사용하고 있습니다. 본 수업의 실습 사례는 린 디자인 프로세스를
따라 진행되며, 디자인 의사결정이 필요한 때에 A/B 테스팅을 사용한 정량적
데이터 분석의 결과를 따릅니다. 다만 이 방법이 다른 방법보다 좋아서
선택했다기 보다는 데이터 기반 디자인 방법론과 잘 어울려 사용할 수 있기
때문에 린 디자인 프로세스를 선택한 것입니다. 그래서 본 교재를 통하여
린(Lean) 디자인 프로세스와 A/B 테스팅을 사용한 정량적 데이터 분석 방법을
주로 학습한다고 할 수도 있을 것 같습니다.

\begin{itemize}
\tightlist
\item
  개선 디자인 프로젝트에 대한 디자인 리서치 포트폴리오
\end{itemize}

제공하는 전체 학습 과정을 정리 요약하면, 기존 서비스의 개선 디자인
프로젝트에 대한 디자인 리서치 포트폴리오를 완성할 수 있습니다. 본 실습
프로젝트는 기존에 운영 중인 서비스를 점진적, 반복적, 정량적으로 개선하는
과정을 실습하므로, 디자인 결과물의 참신성이나 창의성보다는 데이터 기반
디자인과 A/B 테스팅 기법을 활용한 방법론적 측면이 프로젝트의 차별점으로
보이게 될 것입니다. 그래서 본 교재의 학습 내용을 모든 디자인에 적용
가능한 일반적 방법론으로 보기 보다는 서비스 운영 중에 점진적인 디자인
개선을 목표로 하는 디지털 서비스에 적용하기 좋은 방법으로 인식하고,
이러한 특성에 맞는 실습 과제를 선택해서 진행해보는 것이 독자들에게
유용할 것으로 생각됩니다.

\section{이 책에서 다루지 않는
것:}\label{uxc774-uxcc45uxc5d0uxc11c-uxb2e4uxb8e8uxc9c0-uxc54auxb294-uxac83}

이 책에서는 디자인 교재라면 마땅히 포함되어 있을 법한 그래픽 디자인 관련
주제들은 다루지 않습니다. 브랜딩이나 캐릭터 디자인, 모션 그래픽과 같은
디자인 구성 요소를 디자인 해결안의 방향으로 사용하지 않습니다. 그 이유는
당연히 그런 디자인 방향이 의미가 없다는 뜻이 아니고, 이 교재에서는 주로
사용자 경험과 정보 구조, 정보의 레이아웃 등 UX, UI 디자인의 문제들을
발견하여 디자인 문제 해결을 하고자 하기 때문입니다. 그리고 UX 디자인에서
가장 중요한 방법론으로 인식되는 대면 사용자 연구도 다루지 않습니다.
사용자의 서비스 활용 현장에서 사용자를 관찰하고, 인터뷰하는 Contextual
Inquiry Interview를 시행하지 않고, 대신 이미 수집된 사용자의 사용 경험
관련 데이터를 분석하고, 기존 서비스의 디자인 구성 요소들에 대한 분석을
실시합니다. 이 부분도 역시 대면으로 사용자 연구를 하는 것이 필요한
상황이 많이 있지만 교재의 학습 범위에 해당하지 않아서 시행하지 않는
것이므로, 마치 디자인 방법중에 데이터 기반 디자인만 하면 된다는 의도로
구성한 것이 아님을 이해해야합니다.

또한 독자들의 예상과 달리 데이터 분석을 위한 프로그래밍 학습도 심도있게
다루지 않습니다. 이 책은 디자인 방법론을 학습하는 책이므로 데이터 기반
디자인 방법론의 이해와 실습에 초첨을 맞추고 있습니다. 만약 이 책을
통하여 R이나 파이썬과 같은 데이터 분석을 위한 프로그래밍 기술과 분석
방법에 관심이 생겼다면 구체적인 데이터 문해력 학습은 다른 기회에
체계적으로 공부해야합니다. 그래서 이 책은 데이터 문해력이 갖추어진
학습자에게는 빠르게 학습할 수 있는 경로를 제공하고, 데이터 문해력이 낯선
학습자에게는 데이터 분석 기술이 부족한 상태에서도 진행할 수 있는 경로를
제공할 뿐 아니라, 향후 데이터 분석에 관심을 갖고 공부할 수 있도록
안내해주는 데 목적이 있습니다.

아무래도 책을 시작하려는 독자들 스스로 맞는 선택을 하고 있는지를
확인하도록 설명하려다 보니 말이 길어지고 있는데요, 독자 여러분들의
필요에 맞게 본 교재를 선택 하고, 경험하고 싶었던 내용들을 얻어 가셨으면
하는 바램입니다. 그럼 이후의 학습 내용들을 잘 활용하셔서 데이터 기반
디자인을 잘 수행할 수 있는 디자이너의 역량을 갖추시기 바랍니다.

\bookmarksetup{startatroot}

\chapter{1-1. UX 디자인의 오늘과 변화의
방향}\label{ux-uxb514uxc790uxc778uxc758-uxc624uxb298uxacfc-uxbcc0uxd654uxc758-uxbc29uxd5a5}

본 교재가 UX 디자인의 방법론으로서의 데이터 기반 디자인과 A/B 테스팅을
주제로 다루고 있으므로, 이 주제가 UX 디자인의 전체 모습에서 어떻게
위치하고 있는지를 이해하기 위하여 UX 디자인의 큰 그림을 한 번 조망해 볼
필요가 있다고 생각됩니다. 이 큰 그림을 제가 가진 제한된 경험 안에서
스스로 그려내기는 어려울 것 같고, 오랜 시간동안 제가 근간으로 여겨 온 한
권의 책을 기준으로 책 내용의 목차 변화를 관찰해 보면서 서술하고자
합니다. Interaction Design - beyond Human-Computer Interaction, 이 책은
2002년 첫 판이 나온 이후 업데이트를 지속하여 2023년 6판이 나와있는
책인데, 제목에 UX 디자인이라는 말이 들어있지도 않고, 저자들이 디자이너도
아니지만, 오랜 시간 UX디자인의 지식 변화를 교과서 형식의 책으로
담아왔다는 점에서 관심있게 볼 만한 지식 체계라고 생각합니다.(1) 이 책의
모든 개정판을 읽어온 독자로서 발견한 UX 디자인의 키워드 변화들을
중심으로 UX 디자인이 어떤 방향을 바라보고 있는지를 정리해 보겠습니다.

\subsection{Switching to Digital (디자인 대상의
변화)}\label{switching-to-digital-uxb514uxc790uxc778-uxb300uxc0c1uxc758-uxbcc0uxd654}

1990년대 중반 이전에는 복잡하고 기능이 많은 전자 기기들이 대표적인 UX
디자인의 대상으로 인식 되었다면, 그 대상은 인터넷을 플랫폼으로 하는 웹과
모바일 디바이스에서 구현되는 앱 서비스로 진화해왔고, 이들은 또 플랫폼의
경계가 없거나, 매우 다양한 플랫폼에서 제공되는 서비스 및 가상 세계의
사용자 경험으로, 나아가 우리가 경험해 온 물리적 세계에서는 존재하지
않았던 인공지능과 로봇, 그 외의 다양한 디지털 존재들이 공존하는 세계에
대한 디자인으로 확장되고 있습니다. 이러한 사용자 경험 확대 현상의
배경에는 점점 더 커지고 있는 디지털 데이터와 디지털 서비스 기술이
있습니다. 시간과 공간, 물리적 한계를 넘어서는 경험에 대한 UX 디자인은
계속 새로이 창조되고 있으며, 기술의 발전을 바탕으로 그 다양성과 속도가
가속화 되는 중입니다. UX 디자인은 사용자들의 눈높이에 맞는 경험을
제공하는 번역가의 역할에서, 새로운 경험을 창조하고 소개하는 소설가 같은
역할로 변모하고 있습니다. 우리는 현재 시점의 사용자 경험의 영역에
머무르지 말고, 아직 발견되지 않은 새로운 경험의 영역에 대하여 열린
마음을 갖고 탐험할 준비가 되어 있어야 하겠습니다.

\subsection{from User to People (서비스를 사용하는 주체의
변화)}\label{from-user-to-people-uxc11cuxbe44uxc2a4uxb97c-uxc0acuxc6a9uxd558uxb294-uxc8fcuxccb4uxc758-uxbcc0uxd654}

보통 UX 디자인을 제공하는 대상은 사용자(User)로 부르고, 사용자 중심의
디자인(UCD :User Centered Design)은 사용자 경험 디자인의 가장 기본적인
패러다임으로 인식되었습니다. 그런데 이 사용자의 개념이 확장되고
있습니다. 이 확장된 사용자는 사람(People)으로 부릅니다. (이것 참 People
번역을 어떻게 하는게 적절할지 모르겠네요.) 'People'은 인지 과학자이며,
UX 디자인 연구자인 도널드 노먼(Don Norman)이 2018년 제시한 용어로, 개별
사용자 뿐 아니라 여러 사람이 모인 집단, 나아가 사회 구성원들을 의미하는
단어로 소셜 미디어 같이 많은 사람들이 참여하는 커뮤니티를 포함하는 보다
넓은 사용자의 개념입니다. 사람 중심의 디자인(People Centered Design)은
기존의 개별 사용자, 또는 특정 속성을 가지는 사용자 집단의 니즈(Needs)와
역량에 맞추어 사용자 경험을 제공하는 서비스 보다는 더 넓은 범위인 일반
사용자들의 경험과 신체적, 인지적, 사회적 제약이 있는 사용자들도 누구나
참여할 수 있는 사용자 경험을 지향합니다. 그만큼 UX 디자인을 제공하는
대상이 넓어지고, 대중화 되었다는 의미이기도 하고, 어떤 서비스의 경험을
디자인 할 때, 서비스를 직접 사용하는 사용자의 범위가 넓어진 만큼이나 그
서비스와 직접, 또는 간접적으로 관련된 사람들도 다양하고 많아졌다는
의미입니다.

예를 들어, 디지털 금융 서비스는 각 개별 사용자의 금융 정보에 대한
서비스로서, 어떤 디바이스를 사용하는 지에 상관없이 사용 가능해야하며,
신체적 제약이 있는 사용자에게도 문제없이 서비스를 제공해야 합니다. 또
이것은 은행 사용자가 고객 사용자들의 금융 정보를 관리하는 서비스이고, 각
은행의 금융 정보는 통합, 또는 재구성 되어 제 3의 사용자를 통하여 새로운
금융 정보를 생성하기도 합니다. 그리고 이 정보들은 각 개인들 뿐 아니라,
세금 수납과 보험료 납부와 같은 행정 업무를 바탕으로 여러 공공 기관을
비롯한 기업 사용자들과도 연계되어 있습니다. ATM 단말기 같은 디바이스에서
한 개인의 계좌에 입력된 단편적인 금융 정보는 수많은 금융 관련 시스템에서
다양한 시각으로 해석되고, 재생산됩니다. 이렇게 현재의 사용자의 개념은
작용 대상이 넓고, 다양한 시각을 가지는 주체들의 총합으로 이해되어야
합니다.

그리고 많은 서비스의 경우 사용자의 국적이나 언어, 거주지 등, 물리적인
제약과 관련 없이 사용되고, 사용 상황이 다른 여러 사용자간의 상호 협력을
지원하며, 심지어는 인간 사용자가 아닌 기계(인공지능)와의 협력 방법들도
제공하고 있습니다. 이렇게 넓고 다중적인 사용자 경험을 제공하기 위하여
디자이너는 다양한 사용자 요구를 이해하고 반영할 수 있는 역량을 준비해야
하겠습니다.

\subsection{Data Driven Design (디자인 방법론의
변화)}\label{data-driven-design-uxb514uxc790uxc778-uxbc29uxbc95uxb860uxc758-uxbcc0uxd654}

UX 디자인의 도구나 방법의 영역에서는 데이터의 활용이 중요한 키워드로
떠오르고 있습니다. 사용자들이 디지털 서비스를 사용하면서 남기는 사용자
행동의 흔적들은 거대한 데이터가 되어 사용자 경험의 내용을 통계적으로
측정할 수 있도록 합니다. 지금까지 디자인 리서치에 주로 사용해온 대면
조사, 사용 현장 중심의 조사 방법론들이 소수의 사용자들을 대상으로한 질적
연구였다면, 데이터를 이용한 디자인 방법론은 대규모 사용자를 대상으로
통계적 분석과 사용자 경험의 통계적 모델링을 가능하게 하는 연구이며,
지속적인 데이터 축적을 통하여 서비스 변화에 따른 결과 예측이 가능하고,
다양한 협업 부서와 연계하여 데이터의 활용도를 높이는 방법론입니다.
이제는 디지털 디바이스들을 통하여 데이터를 축적하고, 이 데이터를
활용하여 사용자 경험을 이해하는 방법, 데이터 중심의 디자인 프로세스,
데이터 기반의 다양한 협업 방법, 데이터 기반의 디자인 평가와 예측 기법
등이 UX 디자인 방법론의 영역으로 들어왔습니다. 이 책은 이러한 데이터와
UX 디자인의 연결 지점들을 검토하고, 디자이너의 시각에서 데이터를
이해하고 관리하는 방법을 실행해보는 실습 내용을 제공합니다. 이 책의
안내를 바탕으로 독자 여러분의 디자인 학습과 실행의 과정에 데이터 기반
디자인 방법들이 잘 활용될 수 있기를 바랍니다.

아래의 기사에서는 AI가 개별 사용자의 데이터를 바탕으로 각 사용자에게
맞는 UI를 자동적으로 실시간 생성하여 제공하는 Generative UI 시대의
도래를 예고하고 있습니다. 서비스의 UX/UI를 디자이너가 미리 만들어 두고
예상 경로대로 제시하는 것이 아니라, 사용자의 니즈와 사용 데이터에 따라
각 사용자에게 필요한 서비스를 실시간으로 구성해내는 것입니다. 이렇게
된다면 디자이너는 사용자의 니즈를 매우 다른 방법으로 측정하고 디자인하게
될 것입니다. (2)

\href{https://www.nngroup.com/articles/generative-ui/}{Generative UI and
Outcome-Oriented Design}

\subsection{참고 교재의 목차로 알아보는 UX Design의
키워드}\label{uxcc38uxace0-uxad50uxc7acuxc758-uxbaa9uxcc28uxb85c-uxc54cuxc544uxbcf4uxb294-ux-designuxc758-uxd0a4uxc6ccuxb4dc}

다음의 내용은 본 교재의 학습 내용에 대한 것은 아니지만, UX Design의 지식
체계에 대한 큰 그림을 키워드로 볼 수 있다고 생각되어 앞에서 소개한
참고도서인 Interaction Design의 목차를 소개합니다.(1) 목차의 내용을
보면서 어떤 키워드들을 공부해하는지 확인해 보세요. 특히 소제목의
키워드들을 보면 최근의 UX 디자인의 이슈들을 잘 체감할 수 있습니다. (웹
버전은 드롭 다운으로 소제목이 보이게 하기)

\begin{itemize}
\item
  ch1. What is Interaction Design?

  1.1 Introduction

  1.2 Good and Poor Design

  1.3 Switching to Digital

  1.4 What to Design

  1.5 What Is Interaction Design?

  1.6 People-Centered Design

  1.7 Understanding People

  1.8 Accessibility and Inclusiveness

  1.9 Usability and User Experience Goals
\item
  ch2. The process of Interaction Design?

  2.1 Introduction

  2.2 What Is Involved in Interaction Design?

  2.3 Some Practical Issues
\item
  ch3. Conceptualizing Interaction

  3.1 Introduction

  3.2 Conceptualizing Interaction

  3.3 Conceptual Models

  3.4 Interface Metaphors

  3.5 Interaction Types

  3.6 Paradigms, Visions, Challenges, Theories, Models, and Frameworks
\item
  ch4. Cognitive Aspects

  4.1 Introduction

  4.2 What Is Cognition?

  4.3 Cognitive Frameworks
\item
  ch5. Social Interaction

  5.1 Introduction

  5.2 Being Social

  5.3 Face-to-Face Conversations

  5.4 Remote Collaboration and Communication

  5.5 Co-Presence

  5.6 Social Games
\item
  ch6. Emotional Interaction

  6.1 Introduction

  6.2 Emotions and Behavior

  6.3 Expressive Interfaces: Aesthetic or Annoying?

  6.4 Affective Computing and Emotional AI

  6.5 Persuasive Technologies and Behavioral Change

  6.6 Anthropomorphism
\item
  ch7. Interfaces

  7.1 Introduction

  7.2 Interface Types

  7.3 Natural User Interfaces and Beyond

  7.4 Which Interface?
\item
  ch8. Data Gathering

  8.1 Introduction

  8.2 Six Key Issues

  8.3 Capturing Data

  8.4 Interviews

  8.5 Questionnaires

  8.6 Observation

  8.7 Putting the Techniques to Work
\item
  ch9. Data Analysis, Interpretation and Presentation

  9.1 Introduction

  9.2 Quantitative and Qualitative

  9.3 Basic Quantitative Analysis

  9.4 Basic Qualitative Analysis

  9.5 Analytical Frameworks

  9.6 Tools to Support Data Analysis

  9.7 Interpreting and Presenting the Findings
\item
  ch10. Data at Scale, and Ethical Concerns

  10.1 Introduction

  10.2 Approaches for Collecting and Analyzing Data

  10.3 Visualizing and Exploring Data

  10.4 Ethical Design Concerns
\item
  ch11. Discovering Requirements

  11.1 Introduction

  11.2 What, How, and Why?

  11.3 What Are Requirements?

  11.4 Data Gathering for Requirements

  11.5 Bringing Requirements to Life: Personas and Scenarios

  11.6 Capturing Interaction with Use Cases
\item
  ch12. Design, Prototyping and Construction

  12.1 Introduction

  12.2 Prototyping

  12.3 Conceptual Design

  12.4 Concrete Design

  12.5 Generating Prototypes

  12.6 Construction
\item
  ch13. Interaction Design in Practice

  13.1 Introduction

  13.2 AgileUX

  13.3 Design Patterns

  13.4 Open Source Resources

  13.5 Tools for Interaction Design
\item
  ch14. Introducing Evaluation

  14.1 Introduction

  14.2 The Why, What, Where, and When of Evaluation

  14.3 Types of Evaluation

  14.4 Evaluation Case Studies

  14.5 What Did We Learn from the Case Studies?

  14.6 Other Issues to Consider When Doing Evaluation
\item
  ch15. Evaluation studies: from Controls to Natural Settings

  15.1 Introduction

  15.2 Usability Testing

  15.3 Conducting Experiments

  15.4 In-the-Wild Studies
\item
  ch16. Evaluation: Inspections, Analytics and Models

  16.1 Introduction

  16.2 Inspections: Heuristic Evaluation and Walk-Throughs

  16.3 Analytics and A/B Testing

  16.4 Predictive Models
\end{itemize}

\href{http://id-book.com}{Interaction Design - beyond Human-Computer
Interaction}

\subsection{실습과제 1 : 내 디자인 프로젝트
리뷰}\label{uxc2e4uxc2b5uxacfcuxc81c-1-uxb0b4-uxb514uxc790uxc778-uxd504uxb85cuxc81duxd2b8-uxb9acuxbdf0}

\begin{itemize}
\item
  본인이 경험한 디자인 프로젝트 2개를 선정하여 디자인 내용과 결과물을
  요약 정리해보세요. (1 page) 그리고 각 프로젝트의 과정과 방법을
  다이어그램으로 표현합니다(2 page). 각 프로젝트의 목표와 상황에 맞게
  디자인 프로세스와 방법이 수행 되었는지를 스스로 평가하고, 개선 방향이
  있으면 제시합니다.(3 page)

  이 과정을 통하여 현재 본인이 경험해본 디자인 내용과 디자인 과정,
  방법에 대하여 이해하고, 본인의 포트폴리오가 어떻게 평가자의 시각에서
  평가 될지를 검토해볼 수 있으며, 향후에 어떤 디자인 프로젝트를 수행할지
  계획할 수 있습니다.

  page 1: 프로젝트 개요, 디자인 콘셉트, 디자인 해결안 포함

  page 2: 프로젝트의 Process, Methodology diagram

  page 3: 프로젝트의 디자인 과정과 방법론에 대한 평가, 향후 계획

  각 프로젝트 당 3페이지 이내의 요약 보고서를 제작 (총 6페이지 이내)
\item
  (자율학습) 참고 교재의 목차 안내를 보면서 각 키워드에 대하여 얼마나
  이해하고 있는지 확인하고, 개인적으로 보강해야할 키워드가 있으면 공부해
  봅니다. 본 교재에서는 예시 목차의 8장에서 13장의 내용들, 그리고 16장과
  관련하여 공부합니다.
\end{itemize}

\subsection{실습과제 1 Check
Point!}\label{uxc2e4uxc2b5uxacfcuxc81c-1-check-point}

본인이 수행한 연습과제1의 결과물을 아래의 시각으로 평가해 봅시다.

\begin{itemize}
\tightlist
\item
  page 1에서 프로젝트의 내용을 잘 설명하고 있는가?

  \begin{itemize}
  \tightlist
  \item
    프로젝트 개요에서 프로젝트의 핵심을 설명하고 있는가.
  \item
    디자인 콘셉트가 디자인 해결안을 결정할 가치있는 내용을 담고 있는가.
  \item
    프로젝트의 목표와 범위에 맞는 디자인 해결안을 제시하였는가.
  \end{itemize}
\item
  page 2에서 디자인 프로세스와 방법론을 잘 분석하였는가?

  \begin{itemize}
  \tightlist
  \item
    프로젝트에 디자인 프로세스 (더블다이아몬드 모델 등)를 어떻게
    반영하였는지 설명하는가.
  \item
    디자인 프로세스에 포함된 여러 디자인 활동들의 연계가 잘 논리적으로
    합당한가.
  \item
    여러 리서치 방법들을 적용했다고 좋은 리서치가 아님. 목적에 맞는
    리서치 방법론을 통하여 가치있는 디자인 인사이트 발견이 있어야함.
  \end{itemize}
\item
  page 3에서 이상의 관점으로 본인의 프로젝트를 분석하고, 본인이 사용한
  디자인 방법론에 대한 장단점을 서술함. 단점이 발견된 경우 어떻게 보강할
  것인지를 서술함.
\end{itemize}

(문헌1) Yvonne Rogers, Helen Sharp, Jennifer Preece, 『Interaction
Design: Beyond Human-Computer Interaction 6th edition』, Wiley. Kindle
Edition, (2023)

(문헌2) Kate Moran, Sarah Gibbons, ``Generative UI and Outcome-Oriented
Design'', \url{https://www.nngroup.com/articles/generative-ui/}, (2024)

\bookmarksetup{startatroot}

\chapter{1-2. UX 디자인과 데이터,
AI}\label{ux-uxb514uxc790uxc778uxacfc-uxb370uxc774uxd130-ai}

\subsection{데이터의 시각으로 해석한
디자인}\label{uxb370uxc774uxd130uxc758-uxc2dcuxac01uxc73cuxb85c-uxd574uxc11duxd55c-uxb514uxc790uxc778}

UX 디자인에서 데이터를 사용하는 것은 새로운 현상이 아닙니다. 디자인
과정에서 데이터는 언제나 사용되어 왔습니다. 우리가 수행하는 디자인
리서치는 대부분 디자인 주제와 관련한 데이터를 수집하는 것으로
시작됩니다. 그리고 디자이너는 수집한 데이터를 검토하여 필요한 데이터를
선정하고, 선정된 데이터를 사용하여 디자인에 적용할 수 있는 형태의
의미있는 데이터로 변환합니다. 디자인에 적용할 수 있는 형태의 데이터에는
설문이나 관찰 조사된 데이터의 분석 보고서도 있고, 데이터의 내용을
시각적으로 표현한 그래프나 다이어그램, 디자이너가 추출한 인사이트,
그리고 디자이너가 제작한 문제의 구조 다이어그램도 있을 것입니다. 또한
디자인 콘셉트를 표현한 스케치나 도면도 데이터가 됩니다. 그리고 이러한
의미있는 디자인 데이터들은 디자이너의 의도에 의하여 목표하는 디바이스나
플랫폼에서 작동하는 화면 레이아웃이나 버튼, 인터렉션을 발생하는 이벤트
등의 디자인 구현 데이터들로 변환되어 디지털 서비스의 개발 코드에
사용됩니다. 결국 데이터의 시각에서 디자인 과정을 해석하면, 디자인이란
디자인 주제와 관련한 다양한 데이터 재료들을 적절한 방법으로 변환하는
과정을 통하여 디자인 해결안을 구축하는 구현 데이터를 생성하는 과정이라고
말할 수 있습니다.

다만 이 과정에서 사용하는 데이터는 내용과 형식면에서 매우 다양하고, 넓은
범위에 있어서, 데이터들을 선정하고, 해석하고, 필요한 형태로 전환하는
작업들이 디자이너의 주관적이고 창의적인 활동 역량에 의존적이라는 특징이
있습니다. 이와같이 기존의 디자인 방법론도 데이터를 사용하여 디자인을
해왔지만, 최근 주목받는 데이터 기반 디자인(Data Driven Design)의 다른
점은 디자인에 사용하는 데이터들을 디지털화 함으로써 대규모의 데이터를
소프트웨어적인 방법으로 통합, 변형하거나 관리하는 것이 가능한 점,
다양하고 생산성 높은 과학적 분석 기법들을 적용할 수 있는 점, 디자인
내부에서 뿐 아니라 여러 관련 분야의 조직들이 협업하여 이 데이터들을
공유하고 재구성할 수 있다는 점이 다릅니다. 데이터 기반 디자인은 디자인
활동의 내용 측면에서는 기존의 디자인 방법과 동일하지만, 프로젝트에
사용하는 데이터를 수집하고, 가공하고, 활용하는 방법에 변화가 따르게
됩니다.

\subsection{데이터와 AI의
관계}\label{uxb370uxc774uxd130uxc640-aiuxc758-uxad00uxacc4}

최근 디자인을 위한 데이터들이 관심을 받게 된 배경에는 인공지능의 출현이
있습니다. 인공지능 서비스의 본질은 서비스의 제공에 필요한 데이터를
수집하고 이를 인간의 사고 과정과 유사한 데이터 모델로 처리하여 결과를
목표하는 데이터 형식으로 출력하는 과정, 즉 인간이 원하는 가치를 얻어내기
위하여 데이터의 입력, 처리, 출력의 라이프사이클을 인공지능을 사용하여
관리하는 기술이라고 할 수 있는데, 서술하는 바와 같이 인공지능 서비스를
가능하게 하는 기본 재료는 '데이터'입니다. 인간의 사고 과정과 유사한
데이터 모델(인공지능)을 구축하기 위하여는 인간이 데이터를 처리하는
방법을 다량의 데이터로 학습하는 과정이 필요합니다. 이러한 학습 과정을
통하여 인공지능은 인간처럼 대화하고, 판단하고, 예측할 수 있게 됩니다.
그리고 학습 데이터의 질과 양에 따라서 인공지능의 능력이 달라지므로,
인공지능의 역량을 높이기 위해서 양질의 데이터를 다량으로 학습시키고,
테스트하는 과정이 동반됩니다. 그러므로 어떤 인공지능 서비스를 개발할
때는 어떤 데이터를 어떻게 학습시켜서 우리가 원하는 능력을 갖는 인공지능
모델을 구축할 지에 대한 기술적인 이해가 필요하고, 인공지능 서비스를
개발하는 디자이너라면 사용자, 인공지능 데이터 모델, 구현 서비스 간의
통합적인 접점을 제공하는 UX디자인을 제시할 수 있어야 합니다. 기존에
사용자와 구현 서비스간의 UX를 디자인하는 일에 인공지능 데이터 모델이라는
요소가 추가되어 더욱 다면적인 UX 디자인이 요구되는 것입니다.

{[}그림1{]} 다이어그램 추가 필요 (데이터, UX디자인, 서비스, AI)

\subsection{UX 디자인 대상으로서의 AI와 UX 디자인 도구로서의
AI}\label{ux-uxb514uxc790uxc778-uxb300uxc0c1uxc73cuxb85cuxc11cuxc758-aiuxc640-ux-uxb514uxc790uxc778-uxb3c4uxad6cuxb85cuxc11cuxc758-ai}

본 교재의 학습 주제는 데이터 기반 디자인이므로 UX 디자인과 AI의 두가지
연결 지점에서 사용하는 데이터의 특징을 살펴보겠습니다. 첫번째 연결
지점은 인공지능 서비스가 UX 디자인의 대상이 되는 경우, 즉 인공지능
서비스 개발을 위한 UX 디자인을 하는 지점입니다. 이 경우에 디자이너가
다루는 데이터는 인공지능 서비스가 학습하고, 테스트하고, 수집하는 사용자
및 서비스 관련 데이터가 디자인의 대상이 됩니다. 보통 디자이너는 인공지능
서비스 개발에서 사용자가 접하는 화면 그래픽(GUI)이나 인터렉션을
디자인하는 업무에 한정된다고 생각하기 쉽지만, 인공지능 서비스가 어떤
사용자 경험을 수집하고 이해해야 효과적인 서비스가 가능한지를 판단하려면
인공지능 모델 개발의 초기 단계부터 UX 디자이너가 참여해야합니다.

디자이너는 인공지능 모델을 구축할 때 인공지능이 학습할 데이터의 내용과
형식에 대하여 이해하고, 서비스의 환경과 사용자 상황에 맞는 학습 데이터를
선정하는 일에 참여합니다. 또한 인공지능 모델이 구축된 후에 성능을
테스트하고, 서비스의 완성도를 검증하는 일에도 참여해야합니다. 그리고
인공지능 서비스가 제공될 때, 사용자가 서비스와 잘 소통하고, 서비스
내용을 이해할 수 있는 사용자 접점을 디자인해야 합니다. 마지막으로
사용자의 피드백데이터를 기반으로 인공지능 모델과 서비스를 개선할 수 있는
디자인을 제시하여 서비스의 성장을 이끌게 됩니다. 이렇게 UX 디자이너는
인공지능 서비스 개발과 운영, 평가의 전반에 사용자 경험을 반영하여
서비스의 질을 확보하는 중요한 역할을 하고, 이를 위하여 서비스 전체
과정에서 사용되고, 생성되는 데이터에 대한 문해력이 필요하고, 인공지능
모델 개발자, 서비스 개발자 등 기술 전문가와의 협업 능력도 요구됩니다.
마찬가지로, 디자이너와의 효과적인 협업을 위하여 기술 전문가들도 사용자
경험과 디자인을 이해하는 역량이 요구되고 있습니다.

\begin{figure}[H]

{\centering \includegraphics{1-2 UX 디자인과 데이터, AI e536b52818b34fc69680f8f78a377d76/\%E1\%84\%89\%E1\%85\%B3\%E1\%84\%8F\%E1\%85\%B3\%E1\%84\%85\%E1\%85\%B5\%E1\%86\%AB\%E1\%84\%89\%E1\%85\%A3\%E1\%86\%BA_2024-08-05_\%E1\%84\%8B\%E1\%85\%A9\%E1\%84\%92\%E1\%85\%AE_12.39.36.png}

}

\caption{{[}그림2{]} 사용자의 청취 기록을 분석하여 'Discover Weekly'와
같은 맞춤형 콘텐츠를 제공하는 글로벌 음악 스트리밍 서비스, Spotify(3)}

\end{figure}%

{[}그림2{]} 사용자의 청취 기록을 분석하여 'Discover Weekly'와 같은
맞춤형 콘텐츠를 제공하는 글로벌 음악 스트리밍 서비스, Spotify(3)

두번째, 인공지능은 UX 디자인을 위한 도구 역할을 할 수 있습니다. 데이터
분석과 시각화, 아이디어 도출, 디자인 레이아웃 제안, 그래픽 시안 제작 등
디자인 프로세스의 여러 활동들이 인공지능 기술을 활용한 디자인 서비스로
제공되고 있고, 디자인하는 인공지능 서비스는 앞으로도 영역을 확장하여
발전해 나갈 것입니다. 인공지능 기술을 활용한 디자인 방법은 디자인의
생산성을 높이는 데 크게 기여하고 있습니다. 인공지능 기술은 디자이너의
사고와 작업 방식을 학습하여 디자이너처럼 데이터를 처리하고 그 결과물을
제안합니다. 이러한 상황에서 디자이너들이 기술을 더 잘 활용하고, 디자인의
역량을 확대하기 위해서는 디자인하는 인공지능 기술들이 데이터를 처리하는
방법을 이해하고 활용할 수 있어야하며, 나아가 인공지능이 더 높은 수준의
디자인 작업을 지원할 수 있도록 디자인 데이터와 디자인 행위를 데이터
기술과 연계하는 방법을 개발할 필요가 있습니다.

아래의 기사는 UX 디자인 업무에 AI 서비스를 사용하는 구체적인 업무 내용과
방법들을 설명하고 있습니다. (4)

\href{https://www.nngroup.com/articles/ai-ux-getting-started/}{AI for
UX: Getting Started}

2024년 7월에는 대표적인 UX/UI 표현 도구인 피그마에서 AI 기술을
적극적으로 활용한 디자인과 개발, 프레젠테이션 기능들을 선보였습니다. (5)

\href{https://help.figma.com/hc/ko/articles/24037640924823-Config-2024에서-발표된-새로운-기능}{Config
2024에서 발표된 새로운 기능}

이렇게 UX 디자인에 인공지능 기술이 들어와 디자이너의 업무를 수행하게
된다면 어떤 일들이 벌어질까요? 인공지능이 디자인을 할 수 있으므로
디자이너의 일자리가 없어질까요? 인공지능이 담당할 수 없는 창의적이고,
모험적인 업무에는 소수의 디자이너가 주도하고, 인공지능이 디자이너의
업무를 대신하는 것이 가능한 단순 반복적 디자인 작업이나 패턴화된 디자인
작업은 사람대신 인공지능이 대체하게 될 것입니다. 또 많은 디자이너들은
인공지능에게 할 일을 지시하거나 인공지능의 결과물을 검수하는 일, 그리고
인공지능이 디자인 업무를 할 수 있도록 디자인 업무를 디지털 데이터
중심으로 변환하고, 패턴화하는 일을 하거나, 인공지능에게 학습 데이터를
제공하는 일들을 하게 될 것입니다. 이렇게 인공지능 기반 디자인 도구들은
디자이너의 업무 내용이나 업무 역량의 변화를 이끌고, 디자인 전문가가 아닌
사람들과 디자인 업무의 협업자들이 인공지능 서비스를 활용하여 쉽게 UX
디자인 문제들을 해결하게 할 것으로 예상합니다. 기술은 본래 의식이
없습니다. 기술이 우리에게 긍정적인 역할을 할지, 부정적인 역할을 할지는
인간의 사용 의도에 따라 달라질 수 있습니다. 디자이너는 중립적인 인공지능
기술이 우리에게 유리한 방향으로 기여하도록 잘 이끌어 가야 하겠습니다.

(문헌 3) Spotify(스포티파이), Apple app store

(문헌 4) Kate Moran, Jakob Nielsen, ``\textbf{AI for UX: Getting
Started'',}
\url{https://www.nngroup.com/articles/ai-ux-getting-started/}, (2023)

(문헌 5) Figma Learn, ``\textbf{Config 2024에서 발표된 새로운 기능'',}
\url{https://help.figma.com/hc/ko/articles/24037640924823-Config-2024\%EC\%97\%90\%EC\%84\%9C-\%EB\%B0\%9C\%ED\%91\%9C\%EB\%90\%9C-\%EC\%83\%88\%EB\%A1\%9C\%EC\%9A\%B4-\%EA\%B8\%B0\%EB\%8A\%A5},
(2024)

\bookmarksetup{startatroot}

\chapter{1-3. 데이터 기반 디자인 (Data Driven
Design)}\label{uxb370uxc774uxd130-uxae30uxbc18-uxb514uxc790uxc778-data-driven-design}

데이터 기반 디자인(Data Driven Design)이란 디자인 대상과 관련된 방대한
정량화 데이터를 디자인 설계와 의사 결정의 근거로 활용하는 기법과 정량화
테스트를 통하여 디자인 결과물을 평가, 검증하는 기법을 말합니다.(6) 쉽게
말하면, 데이터에 근거하여 디자인 의사 결정을 수행하는 기법이라고 할 수
있습니다. 데이터 기반 디자인 방법론은 디지털 데이터들을 지속적으로
생산하고 축적하는 웹이나 앱 기반 서비스에 쉽게 적용되기 때문에 구글 등
IT 서비스 기업들이 디지털 서비스의 개발과 운영에 활용하고 있습니다.
현재의 데이터 기반 디자인 연구들은 대부분 인터넷을 통하여 수집된 사용자
로그 데이터를 대상으로 이루어지고 있으나, 점차 사물인터넷(IoT: Internet
of Things) 센서 데이터를 사용하는 등 더 다양한 데이터 자원을 활용하는
방향으로 발전하고 있습니다.

데이터 기반 디자인 서비스 및 컨설팅을 제공하는 산업은 증가 추세에
있으며, 디지털 콘텐츠(웹/앱 서비스)의 사용자 로그 데이터를 시각화하여
사용자 유입 분석, 사용 행태 분석, A/B 테스팅 분석 결과를 제공하는
서비스가 주를 이룹니다. 데이터 기반 디자인을 수행하기 위하여 정량적 통계
분석 및 데이터 시각화 서비스를 지원하는 도구 및 컨설팅 비즈니스 사례에는
구글 애널리틱스(Google Analytics), 태블로(Tableau), 어도비 타깃(Adobe
Target) 서비스 등이 있으며, 각 서비스에 대하여는 다음 단원에서 자세히
다루도록 하겠습니다. (7)

앞서 인공지능 서비스를 디자인하기 위해서는 디자이너가 데이터에 대한
이해가 풍부해야한다고 설명했는데, 선행 연구에 의하면, 인공지능 서비스
개발을 위하여 디자이너는 서비스 및 사용자 관련 데이터를 원격으로 수집할
수 있는 능력과 수집된 데이터에 대한 정량적이고 정성적인 분석과 시각화를
통하여 사용자 행동을 해석하여 디자인에 반영할 수 있는 능력이 요구됩니다.
그리고 실제로 인공지능 서비스를 개발하는 디자이너들은 데이터 중심의 업무
환경에서 데이터의 기초적 통계 분석, 데이터 시각화, A/B 테스팅 기법을
매우 자주 사용하고 있다고 합니다. 그러므로 연구 팀은 디자이너가 데이터로
생각하고, 데이터로 작업하며, 데이터 전문가들과 소통할 수 있도록 하는
교육 과정을 UX 디자인 교육에 반영해야 한다고 주장하고, 또한 UX, HCI
교육에서 디자이너, 데이터 전문가, 엔지니어, 사용자 연구자 등이 함께
협업하는 학제적 수업을 제공하여 학생들이 협업 능력과 창의성을 기를 수
있도록 해야 한다고 제안하였습니다.(8) 선행 연구에서 제시한 바와 같이,
인공지능 서비스를 디자인하기 위한 데이터 문해력은 데이터 기반 디자인을
수행하기 위해 필요한 역량과 일치하고 있습니다. 데이터 기반 디자인을
공부하면서 인공지능 서비스의 디자인 역량을 함께 준비할 수 있다니, 더욱
기대되지 않나요? 그럼 다음 단원 부터는 구체적으로 데이터 기반 디자인의
방법들을 경험해 볼까요?

(문헌 6) Rochelle King, Elizabeth Churchill, and Caitlin Tan,
『Designing with data』, O'reilly, (2017), pp.3-6

(문헌 7) 이현진, 『데이터 드리븐 디자인』, UX리뷰, (2024), pp.43-44

(문헌 8) Qian Yang, Alex Scuito, John Zimmerman, Jodi Forlizzi, and
Aaron Steinfeld, ``Investigating How Experienced UX Designers
Effectively Work with Machine Learning'', DIS(Design Information
Systems)(2018), Hong Kong, pp.585--596

\bookmarksetup{startatroot}

\chapter*{참고문헌}\label{uxcc38uxace0uxbb38uxd5cc}
\addcontentsline{toc}{chapter}{참고문헌}

\markboth{참고문헌}{참고문헌}

\phantomsection\label{refs}
\begin{CSLReferences}{0}{1}
\end{CSLReferences}




\end{document}
